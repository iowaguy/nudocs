\documentclass{article}

\usepackage{fancyhdr}
\usepackage{extramarks}
\usepackage{amsmath}
\usepackage{amsthm}
\usepackage{amsfonts}
\usepackage{tikz}
%% \usepackage[plain]{algorithm}
\usepackage{algpseudocode}
\usepackage{enumitem}
%% \usepackage[shortlabels]{enumerate}
\usepackage{multicol}
\usepackage{centernot}
%% \usepackage{textcomp}
%% \usepackage{ifthen}
%% \usepackage{algorithmic}

\usetikzlibrary{automata,positioning}

%
% Basic Document Settings
%

\topmargin=-0.45in
\evensidemargin=0in
\oddsidemargin=0in
\textwidth=6.5in
\textheight=9.0in
\headsep=0.25in

\linespread{1.1}

\pagestyle{fancy}
\lhead{\hmwkAuthorName}
\chead{\hmwkClass: \hmwkTitle}
\rhead{\firstxmark}
\lfoot{\lastxmark}
\cfoot{\thepage}

\renewcommand\headrulewidth{0.4pt}
\renewcommand\footrulewidth{0.4pt}
\newcommand{\supe}[1]{\ensuremath{^{\textrm{#1}}}}
\newcommand{\sub}[1]{\ensuremath{_{\textrm{#1}}}}

\setlength\parindent{0pt}

%
% Create Problem Sections
%

\newcommand{\enterProblemHeader}[1]{
    \nobreak\extramarks{}{Problem \arabic{#1} continued on next page\ldots}\nobreak{}
    \nobreak\extramarks{Problem \arabic{#1} (continued)}{Problem \arabic{#1} continued on next page\ldots}\nobreak{}
}

\newcommand{\exitProblemHeader}[1]{
    \nobreak\extramarks{Problem \arabic{#1} (continued)}{Problem \arabic{#1} continued on next page\ldots}\nobreak{}
    \stepcounter{#1}
    \nobreak\extramarks{Problem \arabic{#1}}{}\nobreak{}
}

%% \setcounter{secnumdepth}{0}
\newcounter{partCounter}
\newcounter{homeworkProblemCounter}
\setcounter{homeworkProblemCounter}{1}
\nobreak\extramarks{Problem \arabic{homeworkProblemCounter}}{}\nobreak{}

%
% Homework Problem Environment
%
% This environment takes an optional argument. When given, it will adjust the
% problem counter. This is useful for when the problems given for your
% assignment aren't sequential. See the last 3 problems of this template for an
% example.
%
\newenvironment{homeworkProblem}[1][-1]{
    \ifnum#1>0
        \setcounter{homeworkProblemCounter}{#1}
    \fi
    \section{Problem \arabic{homeworkProblemCounter}}
    \setcounter{partCounter}{1}
    \enterProblemHeader{homeworkProblemCounter}
}{
    \exitProblemHeader{homeworkProblemCounter}
}

%
% Homework Details
%   - Title
%   - Due date
%   - Class
%   - Instructor
%   - Author
%

\newcommand{\hmwkTitle}{Project \#1 Report}
\newcommand{\hmwkDueDate}{December 3, 2018}
\newcommand{\hmwkClass}{CS7610}
\newcommand{\hmwkClassInstructor}{Professor Cristina Nita-Rotaru}
\newcommand{\hmwkAuthorName}{\textbf{Ben Weintraub and Jaison Titus}}

%
% Title Page
%

\title{
    \vspace{2in}
    \textmd{\textbf{\hmwkClass:\ \hmwkTitle}}\\
    \normalsize\vspace{0.1in}\small{Due\ on\ \hmwkDueDate}\\
    \vspace{0.1in}\large{\textit{\hmwkClassInstructor}}
}

\author{\hmwkAuthorName}
\date{}

\renewcommand{\part}[1]{\textbf{\large Part \Alph{partCounter}}\stepcounter{partCounter}\\}

%
% Various Helper Commands
%

% Useful for algorithms
\newcommand{\alg}[1]{\textsc{\bfseries \footnotesize #1}}

% For derivatives
\newcommand{\deriv}[1]{\frac{\mathrm{d}}{\mathrm{d}x} (#1)}

% For partial derivatives
\newcommand{\pderiv}[2]{\frac{\partial}{\partial #1} (#2)}

% Integral dx
\newcommand{\dx}{\mathrm{d}x}

% Alias for the Solution section header
\newcommand{\solution}{\textbf{\large Solution}\\}

% Probability commands: Expectation, Variance, Covariance, Bias
\newcommand{\E}{\mathrm{E}}
\newcommand{\Var}{\mathrm{Var}}
\newcommand{\Cov}{\mathrm{Cov}}
\newcommand{\Bias}{\mathrm{Bias}}

\begin{document}

\maketitle

\pagebreak
% describing the system architectures, state diagrams, design decisions, and implementation issues

%% intro
\section{Introduction}
For our final project, we have chosen to implement a collaborative document editor.

%%context
\section{Context}
The idea of computer supported cooperation work (CSCW) was a popular area of research in the late 1980's and 1990's. Until 1989, two methods were used for sharing documents, 1) locking; and 2) transactions. If locking was used, then when one user went make an edit, a lock was placed on the document to prevent concurrent and possible destructive edits. There are various optimizations for locking, like locking only a part of the document, but in the end, it is not really possible for the system to know when a user is done editing or just pausing and so locking is not a good strategy. Transactions were proposed as a strategy, and can be effective. They have a cost however, because to execute a transaction in a distributed system, there will need to be at least a two phase commit among all participants, and conflicts may need to be resolved by some sort of consensus. While possible, this means that an edit made locally by one user cannot show up in their document until the transaction completes, resulting in unusable latency for document editing software. In Concurrency Control in Groupware Systems (Ellis et al., 1989) an idea is proposed called Operational Transformation. The high level idea is the following. Any edits that are made locally show up immediately in the editor, and then are sent out to the other peers. The other peers perform a transformation on the operation such that the original operation can be applied to a potentially different state on each peer. For instance, imagine two users concurrently delete the first character in a document. If the operation was applied immediately at both sites, and then applied again after receiving the same operation from the other peer, then two characters would have been deleted, when clearly both users only wanted one character deleted. Operational Transformations are one way to solve this problem and guarantee that each peer's view of the document converges to the same state. The same paper proposed an algorithm called dOPT (distributed Operational Transformation), and our goal was to implement this.\\

\section{Our Experience}
Our classmates, Connor and Giorgio, who were working on a similar concept, made the discovery early on that the dOPT algorithm is incomplete. What this means is that in certain situations having to do with concurrent edits at different sites, operations can be transformed incorrectly. This leads to divergence. As a result of this major issue, we were forced to look into the research and find an alternative algorithm to implement. Looking to maintain the peer-to-peer aspect, we chose an algorithm called REDUCE (Sun et al., 1998). The REDUCE algorithm introduces two new ideas are essential to a correct Operational Transformation (OT). The first is what the authors call ``undo/do/redo''. This means that in order to guarantee convergence, an operation must be applied to a document in the same state as the one in which it was generated. So, if two sites have a series of operations that they have done locally \textit{after} a particular operation was generated on another peer (according to a vector clock that is passed with each operation), then all of those local operations must be undone, so the remote operation can be transformed and applied, and the the undone local operations themselves must be transformed and reapplied. The other idea is that there are actually two different types of transformation functions that are called in different contexts, they are ``inclusion transformations'' and ``exclusion transformations''. Inclusion transformations are roughly equivalent to the transformation functions in dOPT. They are applied when an operation, \(O_a\), should be transformed against \(O_b\) in such a way that \(O_b\)'s effect on the document is accounted for in \(O_a\). Exclusion transformations are a new kind of transformation that are needed as part of the undo/do/redo scheme. They are applied when an operation, \(O_a\), should be transformed against \(O_b\) in such a way that \(O_b\)'s effect on the document is not included in \(O_a\). This new type of transformation is necessary because it guarantees ``intention preservation''. Intention preservation means that after an operation has been applied on a remote node, the same effect should have happened as on the originating node. From the concurrent delete example above, this would mean that only one of the deletes actually changed the document state--the other would be converted into a nop. These two changes in reduce are necessary and sufficient for correctness.\\

\section{Challenges}
Having decided to implement REDUCE, we came to discover that the algorithm is quite complicated. Each operation is passed around to the other peers which themselves generate a slew of new operations as part of the undo/do/redo and transformations. This created many difficulties in debugging, because even if we were to extract the current history of operations that were applied by each peer, it would not be sufficient to know that the document would be converging, because each peer will have executed differently transformed operations based on their own unique document state at the time messages were received from other peers. Testing our program also had its complications because of the complexity of the algorithm. REDUCE has many execution paths, and some of them depend on the timing of relative operations (in terms of their vector clocks). Forcing the timing to meet the constraints of a particular execution path was non-trivial. The most effective testing method we found was to randomly generate operations, but that is not targeted and meant that we were unable to reproduce bugs by rerunning the testing environment.\\

\section{System Architecture}


%% state diagram

%% design decisions

%% implementation issues
%%%%%%% everything was hard, had we known, would have chosen a CRDT algo instead

%% conclusion

%% architecture


\subsection{Orchestration}

\subsection{Sending Messages}

\subsection{Receiving Messages}

\subsubsection{DataMessage}


\subsubsection{AckMessage}





\subsubsection{SeqMessage}


\subsubsection{MarkerMessage}

\subsubsection{MessageStorage}


\subsection{Taking Snapshots}


\subsection{Printing Results}


\section{State Diagram}
The state diagram is represented below in Figure 1.\\

\begin{figure}[h]
  \centering
  \begin{tikzpicture}[shorten >=1pt,node distance=4cm,on grid,auto]
    \node[state, initial] (init)   {initial};
    \node[state] (exit) [below=of init] {exit};
    \node[state] (sending) [right=of init] {sending};
    \node[state] (receiving) [right=of sending] {receiving};
    \node[state] (delivery) [below=of receiving] {delivery};
    \node[state] (startsnap) [right=of receiving] {\begin{tabular}{c} start \\ snapshot \end{tabular}};
    \node[state] (compsnap) [above=of startsnap] {\begin{tabular}{c} complete \\ snapshot \end{tabular}};
    \path[->]
    (init)
    edge node {invalid args} (exit)
    edge [bend left] node {if sender} (sending)
    edge [bend right] node  {if receiver} (receiving)
    (sending)
    edge [bend right] node {data/seq} (receiving)
    (receiving)
    edge [bend right] node {ack} (sending)
    edge [bend left] node {seq} (delivery)
    edge node {\begin{tabular}{c} count \\ reached\end{tabular}} (startsnap)
    (delivery)
    edge [bend left] node {} (receiving)
    (startsnap)
    edge node {} (compsnap)
    edge [bend right=70] node {marker} (sending)
    (compsnap)
    edge node {} (receiving)
  \end{tikzpicture}
  \caption{State Diagram, marking states and possible transitions.}
  \label{fig:multiple5}
\end{figure}

\section{Test Cases}


\section{Design Decisions}

\section{Implementation Issues}

\section{Conclusion}

\section{Sources}
\begin{enum}
  \list Ellis, Clarence A., and Simon J. Gibbs. "Concurrency control in groupware systems." Acm Sigmod Record. Vol. 18. No. 2. ACM, 1989.
  \list Cormack, Gordon V. "A counterexample to the distributed operational transform and a corrected algorithm for point-to-point communication." University of Waterloo Technical Report (1995).
  \list Sun, Chengzheng, and Clarence Ellis. "Operational transformation in real-time group editors: issues, algorithms, and achievements." Proceedings of the 1998 ACM conference on Computer supported cooperative work. ACM, 1998.
  \list Sun, Chengzheng, et al. "Achieving convergence, causality preservation, and intention preservation in real-time cooperative editing systems." ACM Transactions on Computer-Human Interaction (TOCHI) 5.1 (1998): 63-108.
\end{enum}
\end{document}
